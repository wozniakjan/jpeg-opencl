% Dokumentace k projektu do predmetu GZN
% Autori: David Hromadka
%         Jan Wozniak

\documentclass[a4paper,11pt]{article}
\usepackage[czech]{babel}
\usepackage[utf8]{inputenc}
\usepackage[left=1.5cm,text={18cm, 25cm},top=2.5cm]{geometry}
\usepackage{color}
\usepackage[unicode, colorlinks,hyperindex,plainpages=false,pdftex]{hyperref}
\usepackage{graphicx}
\usepackage{float}
\usepackage{multirow}

\begin{document}

\begin{center}{\LARGE\textbf{Komprese/dekomprese JPEG obrázků pomocí 3D akcelerační karty}}\\[0.2cm]
\newcommand{\autor}[2]{#1&\texttt{#2@stud.fit.vutbr.cz}\tabularnewline}
\begin{tabular}{ll}
    \autor{Lucie Matušová}{xmatus21}
    \autor{Jan Wozniak}{xwozni00}
\end{tabular}
\end{center}


\section{Úvod}
Úkolem projektu bylo vybrat částí algoritmu při kompresi/dekompresi, které jsou vhodné pro paralelizaci, implementovat
a optimalizovat dané části v OpenCL a nakonec porovant rychlost implementace s CPU. Jelikož implementace JPEG kodéru a
dekodéru je velmi pracnou záležitostí\cite{t81}, rozhodli jsme se výkonnosti jednotlivých částí pipeline implementovat
a měřit zvlášť.

Dále jsou v textu popsány vybrané algoritmy a porovnání paralelních variant na GPU se seriovými, které vykonává CPU.

\section{Teoretický rozbor} %asi by bylo vhodne lepe pojmenovat tuto kapitolu
Schéma pipeline pro JPEG kompresi na obrazku \ref{pipeline} obsahuje barevně odlišené bloky, což jsou algoritmy,
které jsme se rozhodli paralelizovat. Mezi jednotlivými bloky jsou uvedeny i datové typy, jaké jsou mezi výstupem
jednoho bloku a vstupem druhého bloku očekávány.

\textit{Převod barevného modelu}

\textit{Diskrétní kosínova transformace} slouží k převodu obrazu z prostorové domény do frekvenční domény. V normě JPEG
je definováno 6 variant, z nichž nejčastěji používaná Baseline DCT, kterou jsme také v rámci projektu implementovali\cite{mul}. 
Po aplikaci kosínové transformace dostaneme dvě složky koeficientů -- stejnosměrnou (AC) a střídavou (DC). Tyto složky
se dále v algoritmu komprimují zvlášť. Vzorce \ref{dct1}, \ref{dct2} a \ref{dct3} jsou matematickým zápisem DCT,
kde platí, že \textit{N} je pro JPEG blok rovno 8.
\begin{eqnarray}
\lambda_x &=& \left\{ \begin{array}{r@{\quad}c}
    \frac{1}{\sqrt{2}}, & x = 0 \\
    1, & x \neq 0\\ 
\end{array} \right.\label{dct1}\\
g_{k,j}[n,m] &=& \lambda_k \lambda_j \frac{2}{N} cos\left[\frac{k \pi}{N}\left(n+\frac{1}{2}\right)\right] cos\left[\frac{j \pi}{N}\left(m+\frac{1}{2}\right)\right]\label{dct2}\\
c[k,j] &=& \sum^{N-1}_{n=0}{\sum^{N-1}_{m=0}{f[n,m]g_{k,j}[n,m]}}\label{dct3}
\end{eqnarray}


\textit{Kvantizace} je implementována pomocí dvou tabulek, zvlášť pro jasovou složku a chrominescenční složky. Hodnota v 
bloku je nejprve vydělena přislušnou hodnotou v tabulce, zaokrouhlena a poté vynásobena stejnou hodnotou z tabulky.
Pro kvantizaci je definováno několik vhodných tabulek\cite{t81}.

\textit{Huffmanovo kódování}

\begin{figure}[H]
  \centering
  \includegraphics[width=15cm]{pipeline.pdf}
  \caption{Schéma JPEG encode pipeline.}
  \label{pipeline}
\end{figure}

\section{Implementace}
Celý projekt je rozdělen do několika modulů
\begin{itemize}
\item \texttt{cl\_util} -- funkce potřebné pro práci s OpenCL, jednotlivé kernely jsou v souboru \texttt{jpeg.cl}.
\item \texttt{color\_transform} -- 
\item \texttt{dpcm\_rle} -- 
\item \texttt{huffman} -- 
\item \texttt{jpeg\_util} -- tvorba a čtení JPEG markerů a další pomocné funkce pro JPEG, které nemají samostaný
modul (zig-zag, kvantizace, DCT ...).
\item \texttt{main} -- funkce \texttt{main} a časová měření algoritmů.
\end{itemize}
\textit{Diskrétní kosínova transformace} je implementována pomocí funkcí

\texttt{void dct8x8(float* block, float* dct, int* table)} -- sériově

\texttt{void dct8x8\_gpu(float* block, float* dst, cl\_mem* table)} -- paralélně\\
kde prvním argumentem jsou data bloku vstupujícího do DCT, druhým argumentem pole pro výstupní AC a DC koeficienty DCT
a třetím argumentem je kvantizační tabulka, neboť v rámci optimalizace jsme sjednotili kvantizaci s DCT.


\section{Závěr}
Výsledkem naší práce je srovnání rychlosti implementací na těchto strojích.
\begin{itemize}
\item Ubuntu -- Core 2 Duo (T7500) 2.20GHz, NVIDIA GeForce 8600M GT
\item Debian -- Core i5 (2500K) 3.3 GHz, NVIDIA GeForce 8800 GTX
\item OS X -- Core i5 (I5-3317U) 1.7 GHz, HD Graphics 4000
\end{itemize}
Hodnoty měření jsou uvedeny v milisekundách, porovnávány jsou paralelní verze a sériové verze algoritmů.

\begin{table}[H]
\begin{center}
\renewcommand{\arraystretch}{1.3} %zvetseni mezer mezi radky v tabulce
\begin{tabular}{|l|cc|cc|cc|}\hline%
\multirow{2}{*}{Algoritmus}&\multicolumn{2}{|c|}{Ubuntu}&\multicolumn{2}{|c|}{Debian}&\multicolumn{2}{|c|}{OS X}\\
          &serial [ms]&paralel [ms]&serial [ms]&paralel [ms]&serial [ms]&paralel [ms]\\\hline%
\texttt{RGB to YCbCr}&447.89&1008.18&3&4&5&6\\
\texttt{YCbCr to RGB}&1&2&3&4&5&6\\
\texttt{Huffman}     &1&2&3&4&5&6\\
\texttt{Inv\_Huffman} &1&2&3&4&5&6\\
\texttt{DCT}         &0.527859&1.49798&0.320911&0.144005&0.410795&0.264168\\
\texttt{Inv\_DCT}    &0.527143&1.14584&0.276089&0.113011&0.396967&0.203848\\
\hline
\end{tabular}
\renewcommand{\arraystretch}{1} %zvetseni mezer mezi radky v tabulce
\end{center}
\caption{Tabulka srovnání doby výpočtu jednotlivých algoritmů.}
\label{ps_history}
\end{table}

%
%
% LITERATURA
% ======================
\newpage
{%
    \renewcommand{\refname}{Literatura} % Text nadpisu thebibliography.
    \newcommand{\bi}[4]{\bibitem{#1}\textit{#2.} #3\\{}$<$\url{#4}$>$}%
    \newcommand{\citdatum}[1][2011-10-08]{$[$cit.~{#1}$]$}%
%
\begin{thebibliography}{MM}%
% Vzor: \bi{label}{Název}{Poznámky.}{http://www.adresa.cz/}%
\bi{t81}{Recommendation T.81}{\citdatum[2012-12-10]}
    {http://www.w3.org/Graphics/JPEG/itu-t81.pdf}
\bibitem{mul}{D. Bařina. \textit{Diskrétní kosínová transformace -- prezentace ke cvičení}. \citdatum[2008-12-08].\\
    $<$\url{http://www.fit.vutbr.cz/study/course-l.php.cs?id=8766}$>$}
\bi{wiki}{Wikipedia, the free encyclopedia}{\citdatum[2012-12-10]}
    {http://en.wikipedia.org/}
\end{thebibliography}}
\end{document}
