% Dokumentace k projektu do predmetu GZN
% Autori: David Hromadka
%         Jan Wozniak

\documentclass[a4paper,11pt]{article}
\usepackage[czech]{babel}
\usepackage[utf8]{inputenc}
\usepackage[left=1.5cm,text={18cm, 25cm},top=2.5cm]{geometry}
\usepackage{color}
\usepackage[unicode, colorlinks,hyperindex,plainpages=false,pdftex]{hyperref}
\usepackage{graphicx}
\usepackage{float}
\usepackage{multirow}

\begin{document}

\begin{center}{\LARGE\textbf{Komprese/dekomprese JPEG obrázků pomocí 3D akcelerační karty}}\\[0.2cm]
\newcommand{\autor}[2]{#1&\texttt{#2@stud.fit.vutbr.cz}\tabularnewline}
\begin{tabular}{ll}
    \autor{Lucie Matušová}{xmatus21}
    \autor{Jan Wozniak}{xwozni00}
\end{tabular}
\end{center}


\section{Úvod}
Úkolem projektu bylo vybrat částí algoritmu při kompresi/dekompresi, které jsou vhodné pro paralelizaci, implementovat
a optimalizovat dané části v OpenCL a nakonec porovant rychlost implementace s CPU. Jelikož implementace JPEG kodéru a
dekodéru je velmi pracnou záležitostí\cite{t81}, rozhodli jsme se výkonnosti jednotlivých částí pipeline implementovat
a měřit zvlášť.

Dále jsou v textu popsány vybrané algoritmy a porovnání paralelních variant na GPU se seriovými, které vykonává CPU.

\section{Teoretický rozbor} %asi by bylo vhodne lepe pojmenovat tuto kapitolu
Schéma pipeline pro JPEG kompresi na obrazku \ref{pipeline} obsahuje barevně odlišené bloky, což jsou algoritmy,
které jsme se rozhodli paralelizovat. Mezi jednotlivými bloky jsou uvedeny i datové typy, jaké mezi nimi proudí.

\section{Implementace}

\section{Závěr}
Výsledkem naší práce je srovnání rychlosti implementací na těchto strojích.
\begin{itemize}
\item Ubuntu -- 
\item Debian -- Core i5 (2500K) 3.3 GHz, NVIDIA GeForce 8800 GTX
\item OS X -- Core i5 (I5-3317U) 1.7 GHz, HD Graphics 4000
\end{itemize}
Hodnoty měření jsou uvedeny v milisekundách, porovnávány jsou paralelní verze a sériové verze algoritmů.

\begin{table}[H]
\begin{center}
\renewcommand{\arraystretch}{1.3} %zvetseni mezer mezi radky v tabulce
\begin{tabular}{|l|cc|cc|cc|}\hline%
\multirow{2}{*}{Algoritmus}&\multicolumn{2}{|c|}{Ubuntu}&\multicolumn{2}{|c|}{Debian}&\multicolumn{2}{|c|}{OS X}\\
          &serial [ms]&paralel [ms]&serial [ms]&paralel [ms]&serial [ms]&paralel [ms]\\\hline%
\texttt{RGB to YCbCr}&1&2&3&4&5&6\\
\texttt{YCbCr to RGB}&1&2&3&4&5&6\\
\texttt{Huffman}     &1&2&3&4&5&6\\
\texttt{Inv\_Huffman} &1&2&3&4&5&6\\
\texttt{DCT}         &1&2&0.320911&0.144005&0.410795&0.264168\\
\texttt{Inv\_DCT}    &1&2&0.276089&0.113011&0.396967&0.203848\\
\hline
\end{tabular}
\renewcommand{\arraystretch}{1} %zvetseni mezer mezi radky v tabulce
\end{center}
\caption{Tabulka srovnání doby výpočtu jednotlivých algoritmů.}
\label{ps_history}
\end{table}

%
%
% LITERATURA
% ======================
\newpage
{%
    \renewcommand{\refname}{Literatura} % Text nadpisu thebibliography.
    \newcommand{\bi}[4]{\bibitem{#1}\textit{#2.} #3\\{}$<$\url{#4}$>$}%
    \newcommand{\citdatum}[1][2011-10-08]{$[$cit.~{#1}$]$}%
%
\begin{thebibliography}{MM}%
% Vzor: \bi{label}{Název}{Poznámky.}{http://www.adresa.cz/}%
\bi{wiki}{Wikipedia, the free encyclopedia}{\citdatum[2012-12-10]}
    {http://en.wikipedia.org/}
\bi{t81}{Recommendation T.81}{\citdatum[2012-12-10]}
    {http://www.w3.org/Graphics/JPEG/itu-t81.pdf}
%
\end{thebibliography}}
\end{document}
